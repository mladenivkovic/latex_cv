\documentclass{muratcan_cv}

\setname{Mladen}{Ivkovic}
\setstreet{Wehntalerstrasse 298}
\setcity{8046 Zürich}
\setcountry{Switzerland}
\setnationality{Swiss}
\setmobile{+41 76 573 52 56}
\setmail{mladen.ivkovic@epfl.ch}
\setposition{PhD Candidate}
\setcompanyname{\'Ecole Polytechnique F\'ed\'erale de Lausanne}
\setbirthday{12. June 1994}
\setlinkedinaccount{https://www.linkedin.com/in/mladen-ivkovic/}
\setgithubaccount{https://github.com/mladenivkovic}
\sethomepage{https://obswww.unige.ch/~ivkovic}
\setthemecolor{cyan} %you can play with color of the template (red is also nice..)

\newcommand{\swift}{\href{https://swift.dur.ac.uk/}{\textsc{swift}}}
\newcommand{\ramses}{\href{https://bitbucket.org/rteyssie/ramses/src/master/}{\textsc{ramses}}}


\begin{document}
%Set variables
%You can add sections, texts, explanations just by copying the style below. Replace the dummy texts "\lipsum[1][x-x]\par" with actual texts.
%Create header
\headerview
\vspace{1ex}
%Sections
%
% Summary
\addblocktext{Summary}{%
    My background is in physics and astrophysics, with heavy emphasis on computational 
    astrophysics and software development for high performance simulations and on-the-fly 
    analysis. I have been developing scientific software intended for use on shared and 
    distributed memorty systems and supercomputers since 2015, and so far, I have worked on 
    halo finding, mergertree building, fluid dynamics (using meshless methods, finite volume 
    methods, and smoothed particle hydrodynamics), and radiative transfer in astrophysical codes 
    that make use of MPI, OpenMP, and QuickSched.
    Now I am looking for new challenges in the development, improvement, optimization, and extension 
    of open source high performance scientific software! \\[.5em] \par
}

%Education
\section{Education and Employment}
    \datedexperience{University of Zurich}{2012 - 2017}
    \explanation{BSc in Physics (major) and Applied Informatics for Scientists (minor)}
    \explanationdetail{\coloredbullet\ %
        Bachelor thesis on 
        ``\href{https://obswww.unige.ch/~ivkovic/homepage/files/work/bachelor_thesis_mivkov.pdf}
            {Halo- and Subhalo Finding in Cosmological N-Body Simulations}'' 
        with Prof. Romain Teyssier\par
    }
    % \explanationdetail{\coloredbullet\ %
    %     Main goal: Enable on-the-fly analysis of cosmological structures with \ramses\ and correct them
    %     based on physical constraints \par
    % }
    \datedexperience{University of Zurich}{2016 - 2018}
    \explanation{MSc in Theoretical Astrophysics and Cosmology}
    \explanationdetail{\coloredbullet\ %
        Master thesis on 
        ``\href{https://obswww.unige.ch/~ivkovic/homepage/files/work/master_thesis_mivkov.pdf}
            {Creating Mock Galaxy Catalogues from Dark Matter Simulations}'' 
        with Prof. Romain Teyssier\par
    }
    % \explanationdetail{\coloredbullet\ %
    %     Main goal: Enable on-the-fly generation of merger trees of cosmological structures  with \ramses\ \par
    % }
    \datedexperience{\'Ecole Polytechnique F\'ed\'erale de Lausanne}{2018 - today}
    \explanation{PhD Candidate in Astrophysics}
    \explanationdetail{\coloredbullet\ %
        Thesis on ``Dwarf Galaxies in the Epoch of Reionization'' with Dr. Yves Revaz and Prof. Anne Verhamme \par
    }
    % \explanationdetail{\coloredbullet\ %
    %     Main goal: Implementing a radiative transfer physics module from scratch into the
    %     task-based code \swift\ and simulation of the reionization of dwarf galaxies in
    %     the early universe
    % }

%
% Software
\section{Software}
    \datedexperience{Simulation}{}%
    \explanationdetail{\coloredbullet\ %
        Contributed to \ramses\ and \swift\ open source astrophysical high performance
        simulation codes. 
    }
    \explanationdetail{\coloredbullet\ %
        Written a \href{https://github.com/mladenivkovic/mesh-hydro}{didactical finite volume hydrodynamics solver}.
    }
    \datedexperience{Visualisation and Analysis}{}%
    \explanationdetail{\coloredbullet\ %
        Contributed to the \href{https://github.com/SWIFTSIM/swiftsimio}{swiftsimio} and 
        \href{https://gitlab.com/revaz/pNbody}{pNbody} python libraries.
    }
    \explanationdetail{\coloredbullet\ %
        Written a python library for 
        \href{https://pypi.org/project/astro-meshless-surfaces/}{visualisation of ``effective surfaces''} 
        in mesh-free hydrodynamics methods
    }
%
% Skills
\section{Skills}
    %
    \newcommand{\skillone}{\createskill{Programming Languages}{
        \textbf{\emph{Experienced:}} \
        Python \cpshalf 
        C \cpshalf 
        Fortran \cpshalf 
        bash \cpshalf 
        LaTeX \ \ 
        \textbf{\emph{Familiar:}} \ 
        C++ \cpshalf 
        Java \cpshalf Mathematica
        }
    }
    %
    \newcommand{\skilltwo}{\createskill{Parallelisation}{
        \textbf{\emph{Experienced:}} \
        MPI \cpshalf 
        OpenMP \cpshalf 
        QuickSched \ 
        \textbf{\emph{Familiar:}} \
        openACC \cpshalf 
        CUDA
        }
    }
    %
    \newcommand{\skillthree}{\createskill{Software Development}{
        \textbf{\emph{Experienced:}} \
         git \cpshalf 
         CLI \ 
        \textbf{\emph{Familiar:}} \
        GitLab CI/CD  \cpshalf
        sphinx
        }
    }
    %
    \newcommand{\skillfour}{\createskill{Languages}{
        \textbf{\emph{Native:}} \ \  
        German \cpshalf 
        Swiss German \cpshalf 
        Serbian \ \ 
        \textbf{\emph{Fluent:}} \ \ 
        English \ \ 
        \textbf{\emph{Intermediate:}} \ \  
        French 
        }
    }
    %
    \createskills{\skillone, \skilltwo, \skillthree, \skillfour}
%
% Publications
\section{Publications}
    \newcommand{\publicationone}{%
        Mladen Ivkovic, Romain Teyssier, 
        \href{https://ui.adsabs.harvard.edu/abs/2022MNRAS.510..959I/abstract}{\textsc{acacia}: a new method to produce on-the-fly merger trees in the \textsc{ramses} code}, 
        Monthly Notices of the Royal Astronomical Society, Volume 510, 
        Issue 1, February 2022, Pages 959–979, 
        \href{https://doi.org/10.1093/mnras/stab3329}{https://doi.org/10.1093/mnras/stab3329}
    }
    %
    \newcommand{\listofpublications}{\publicationone}
    %
    \createbullets{\listofpublications}

%!!!!!!!!!!!!!!!!!!!!!!!!!!!!!!!!!!!!!!!!!!!!
% WATCH OUT FOR THIS IN THE FUTURE
%!!!!!!!!!!!!!!!!!!!!!!!!!!!!!!!!!!!!!!!!!!!!
% \newpage
%
% Talks
\section{Talks}
    \newcommand{\talkone}{%
        SWIFTcon, Durham, November 2019: \textit{``On Meshless Methods in Astrophysics''}
    }
    %
    \newcommand{\talktwo}{%
        RASCAS-in-SPHINX workshop, Geneva, December 2019: \textit{``On Meshless Methods in Astrophysics''}
    }
    %
    \newcommand{\listoftalks}{\talkone, \talktwo}
    %
    \createbullets{\listoftalks}

%
% Teaching
\section{Teaching}
    \datedexperience{Zurich}{2011 - 2016}
     \explanationdetail{\coloredbullet\ %
        \textsc{Private tutor} in physics, mathematics, and German at elementary-, 
        middle-, high-school and university level \par
    }
    \datedexperience{University of Zurich}{2014 - 2018}
    \explanationdetail{\coloredbullet\ %
        \textsc{Teaching Assistant} for basic physics courses, practical courses in 
        physics, and introduction to programming \par
    }
    \datedexperience{\'Ecole Polytechnique F\'ed\'erale de Lausanne}{2019 - today}
     \explanationdetail{\coloredbullet\ %
        \textsc{Teaching Assistant} for MSc level lectures ``Stellar and Galactic 
        Dynamics'' and ``Observational Cosmology'' \par
    }


%
% Outreach
\section{Outreach}
    \datedexperience{University of Zurich}{2015 - 2017}
     \explanationdetail{\coloredbullet\ %
        Participated in university-wide public outreach events as a point of contact 
        for people interested in studying physics \par
    }
    \datedexperience{\'Ecole Polytechnique F\'ed\'erale de Lausanne}{2018 - today}
     \explanationdetail{\coloredbullet\ %
        Participated in university-wide public outreach events as a point of 
        contact for the general public \par 
        % interested in astrophysics and in virtual reality 
        % demonstrations of astrophysical objects and scales \par
    }
     \explanationdetail{\coloredbullet\ %
        Guide on public and private visitations to the Geneva Observatory \par
    }
%
% Grants
\section{Grants}

    \datedexperience{HPC-Europa3 Transnational Access programme}{2021}
    \explanation{awarded travel grant to Leiden (NL) for 9 weeks and 100'000 CPU hours}
    \explanationdetail{\coloredbullet\ %
        to enhance the parallelisation of the radiative transfer module in \swift\ \par
    }
%
% Extra
\section{Extra}
    \datedexperience{Planet5, Zurich}{2012 - 2015}
    \explanation{Volunteer work at the ``Metal Apocalypse'' concert series}
    \explanationdetail{\coloredbullet\ %
        as event manager, booking agent, allrounder, bartender, and photographer. \par
    }
    %
    \datedexperience{Mundwerk Kulturbiotop, Zurich}{2013 - 2022}
    \explanation{Volunteer work at the youth oriented music club}
    \explanationdetail{\coloredbullet\ %
        as event manager, booking agent, bartender, allrounder, and Chef de Bar. \par
    }
    \explanationdetail{\coloredbullet\ %
        Board member 2013 - 2017 \par
    }
%
%Footnote
% \createfootnote
\end{document}
